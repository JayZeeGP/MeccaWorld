%%%%%%%%%%%%%%%%%%%%%%%%%%%%%%%%%%%%%%%%%%%%%%%%%%%%%%%%%%%%%%%%%%%%%%%%%%%%%
%                                                                           %
%  Nombre: Plantilla maestra de documento (informe, memoria, trabajo, etc)  %
%  Autor: Daikrieg ( http://puedescorrer.es)                                %
%  Fecha: 2008                                                              %
%  Licencia: Creative Commons Atribución 3.0                                %
%                                                                           %
%%%%%%%%%%%%%%%%%%%%%%%%%%%%%%%%%%%%%%%%%%%%%%%%%%%%%%%%%%%%%%%%%%%%%%%%%%%%%

%Clase report, es decir, documento largo con capítulos, y por las dos caras
\documentclass[a4paper,12pt,twoside,openright]{report} 

\usepackage{listings}

\usepackage[utf8]{inputenc}
\usepackage[spanish]{babel} %Localización: Español

\usepackage{array} %Paquete útil para trabajar con tablas
\usepackage{rotating} %Tablas giradas

\usepackage{url} %Formateo e hiperenlaces
%Para que los hipervínculos no aparezcan cajeados
\usepackage[colorlinks=true,linkcolor=black,urlcolor=black]{hyperref} 

\usepackage{amssymb} %Diversos símbolos
\usepackage{eurosym} %Saca el símbolo del euro escribiendo \officialeuro{}
\usepackage{verbatim} %Para meter código fuente
\usepackage{fancyhdr}
\usepackage{amsmath} %Paquete matemáticos
\usepackage{graphicx} %Manejo de imágenes
\usepackage{pdfpages} %Inclusión de pdfs a página completa, para incrustados usar figure
\usepackage{float}

%\usepackage{times} %Fuente alternativa a la Computer modern.
\usepackage{wrapfig} %Paquete para girar imagenes

\usepackage{caption3} 
\DeclareCaptionOption{parskip}[]{} 
\usepackage[small]{caption}  %Pies de foto en letra más pequeña


\title {Memoria de Trabajo de Prácticas\\ Mundo Wumpus}
\author{Procesadores del Lenguaje \\ Primer curso del Segundo Ciclo de Ingeniería Informática (2011-2012)\\Escuela Politécnica Superior de Córdoba\\ Universidad de Córdoba \\ Rubén Salado Cid \\José Carlos Garrido Pérez}
\date{\today}


%%%%%%%%%%%%%%%%%%%%%%%
%  Fin del preámbulo  %
%%%%%%%%%%%%%%%%%%%%%%%

\begin{document}

%Primeras consideraciones
\renewcommand{\tablename}{Tabla} %Las tablas no se llaman cuadros
\renewcommand{\listtablename}{Índice de tablas} 
\renewcommand{\sin}{sen} %Los senos se llaman sen


\maketitle %Genera la clásica página de título

\addtolength{\parskip}{\baselineskip} %Líneas más separadas, después del título

%Las presentaciones y disclaimers no llevan numeración
\pagestyle{empty}

\newpage


% Limpia el encabezado en las páginas impares vacías
\makeatletter \def\cleardoublepage{
\clearpage\if@twoside \ifodd\c@page\else%
\hbox{}%
\thispagestyle{empty}% Aquí elimina el estilo del encabezado
\newpage%
\if@twocolumn\hbox{}\newpage\fi\fi\fi} \makeatother


%Numeración del índice en romanos
\pagenumbering{roman}
\setcounter{page}{1} 
\pagestyle{fancy}
\fancyhf{}
\fancyhead[LO]{\leftmark} % En las páginas impares, parte izquierda del encabezado, aparecerá el nombre de capítulo
\fancyhead[RE]{\rightmark} % En las páginas pares, parte derecha del encabezado, aparecerá el nombre de sección
\fancyhead[RO,LE]{\thepage} % Números de página en las esquinas de los encabezados

\renewcommand{\chaptermark}[1]{\markboth{\thechapter. #1}{}} % Formato para el capítulo: N. Nombre
\renewcommand{\sectionmark}[1]{\markright{\thesection. #1}} % Formato para la sección: N.M. Nombre
%Fin configuración de la numeración del índice


\tableofcontents %Añade el índice general...

\addcontentsline{toc}{chapter}{Índice de tablas} %... el índice de tablas
\listoftables

\addcontentsline{toc}{chapter}{Índice de figuras} %... y el de imágenes
\listoffigures


\newpage


%Numeración del cuerpo del documento en arábigos
\pagenumbering{arabic}
\setcounter{page}{0} 
\pagestyle{fancy}
\fancyhf{}
\fancyhead[LO]{\leftmark} % En las páginas impares, parte izquierda del encabezado, aparecerá el nombre de capítulo
\fancyhead[RE]{\rightmark} % En las páginas pares, parte derecha del encabezado, aparecerá el nombre de sección
\fancyhead[RO,LE]{\thepage} % Números de página en las esquinas de los encabezados

\renewcommand{\chaptermark}[1]{\markboth{\thechapter. #1}{}} % Formato para el capítulo: N. Nombre
\renewcommand{\sectionmark}[1]{\markright{\thesection. #1}} % Formato para la sección: N.M. Nombre
%Fin configuración de la numeración del cuerpo

\newpage




%%%%%%%%%%%%%%%%%%%%%%%%%%%%%%%%%%%%%%%%%%%%%%%%%%%%%%%%%%%%%%
%  Ahora toca empezar a escribir, lo más difícil             % 
%  Puedes hacerlo directamente aquí como ya hemos empezado,  %
%  o puedes incluir archivos externos con el comando input,  %
%  como en los anexos listados más abajo.                    %
%%%%%%%%%%%%%%%%%%%%%%%%%%%%%%%%%%%%%%%%%%%%%%%%%%%%%%%%%%%%%%
\chapter{Introducción \label{intro}}
  \section{Descripción del entorno a simular}
  En la presente memoria de prácticas va a comentarse cuáles han sido los pasos realizados para acometer el desarrollo de un 
  intérprete que acepte y ejecute las sentencias necesarias para poder, en dos modos de ejecución distintos, crear niveles del 
  juego ``Mundo Wumpus'' y también para que el usuario pueda jugar los niveles creados, moviendo al personaje por el mapa 
  creado.
  Este entorno deberá, además, permitir la declaración de variables, el uso de esquemas propios de la programación como los 
  esquemas condicionales, bucles de diverso tipo, operaciones lógicas y aritméticas, etc. En la medida de lo posible, deberá 
  recuperarse de los errores e informar al usuario del problema acontecido, así como de las posibles soluciones por las que 
  puede optarse.

  A grandes rasgos, el entorno que se va a simular con el intérprete cuenta con los dos modos siguientes:
  \begin{itemize}
  \item \textbf{Modo de configuración}: denominado \emph{configurationMode}, es el modo en el que se podrán crear y editar los 
  niveles, con total libertad, configurando aspectos tales como el tamaño del tablero, las posiciones de los hoyos, tesoros, 
  Wumpus, etc. También podrá configurarse el número de flechas que se desea que tenga el jugador a su disposición y las 
  posiciones de comienzo y fin del nivel.
  \item \textbf{Modo de aventura}: denominado \emph{adventureMode}, en el cual, el usuario solamente podrá realizar las 
  acciones que están permitidas a un jugador del juego Mundo Wumpus, esto es, mover al 
  aventurero (al que se le ha dado el nombre ficticio de \emph{Mecca}) por el mapa, 
  disparar para intentar matar al Wumpus, saber cuántos tesoros quedan, etc. En este 
  modo, el jugador recibirá además información por parte del intérprete de los elementos 
  que hay en el mapa, dependiendo de la casilla en el que se encuentre y de los 
  resultados de sus movimientos o disparos. Por ejemplo, si se encuentra cerca del Wumpus 
  se le informará de que existe un hedor, para que sepa que está en una casilla cercana. 
  También se informará, en caso de que el juego llegue a su fin, de que ha logrado escapar 
  del nivel con todos los tesoros existentes o, por el contrario, de que ha muerto.
  \end{itemize}

\section{Fases del proceso de generación del intéprete}
  El proceso de generación de este intérprete ha contado con las siguientes fases:
  \begin{enumerate}
   \item Elección del entorno
   \item Diseño del lenguaje
   \item Desarrollo del análisis léxico
   \item Desarrollo del análisis sintáctico
   \item Atributos y funciones auxiliares
   \item Elegir modo de ejecución
  \end{enumerate}

\chapter{Definición del lenguaje diseñado \label{defleng}}
  \section{Sentencias o instrucciones para simular el entorno elegido}
  Como hay dos modos de ejecución, se van a comentar primero las funciones que son comunes a ambos modos:
  \begin{itemize}
   \item getArrows(): debe devolver el número de flechas que tiene el aventurero.
   \item getMecca(): debe devolver la posición de Mecca en el mapa.
   \item getTotalTreasures(): debe devolver el número total de tesoros.
  \end{itemize}
  Las funciones que pueden ser llamadas sólo desde el modo de configuración son:
  \begin{itemize}
   \item setBoardSize(rows, cols): debe hacer que el tamaño del tablero sea el que recibe como parámetro.
   \item setTreasure(row, col): ha de permitir insertar un tesoro, en caso de que sea posible, en la posición que recibe como 
   parámetro.
   \item getTreasure(number): debe devolver la posición del tesoro cuyo número se le pasa como parámetro.
   \item setHole(row, col): debe permitir insertar un hoyo, en caso de que sea posible, en la posición que recibe como 
   parámetro.
   \item getNumberOfHoles(): devolverá el número de hoyos existentes en el mapa.
   \item getHole(number): ha de devolver la posición del hoyo cuyo número se le pasa como parámetro.
   \item setWumpus(row, col): ha de permitir colocar el Wumpus en la posición que se le pasa como parámetro.
   \item getWumpus(): deovlverá la posición del Wumpus.
   \item setStart(row, col): deberá permitir colocar la casilla de comienzo en la posición que se le pasa como parámetro.
   \item getStart(): debe devolver la posición de comienzo del mapa.
   \item setExit(row, col): deberá permitir colocar la casilla de salida en la posición que se le pasa como parámetro.
   \item getExit(): debe devolver la posición de salida del mapa.
   \item incArrows(number): incrementa el número de flechas de Mecca en una cantidad igual a la que reciba como parámetro.
   \item decArrows(number): decrementa el número de flechas de Mecca en una cantidad igual a la que reciba como parámetro.
  \end{itemize}
  Desde el modo aventura pueden llamarse las siguientes funciones:
  \begin{itemize}
   \item Funciones que se usan para el movimiento del aventurero Mecca:
   \begin{itemize}
    \item goUp(): moverá a Mecca una casilla hacia arriba.
    \item goRight(): moverá a Mecca una casilla hacia la derecha.
    \item goLeft(): moverá a Mecca una casilla hacia la izquierda.
    \item goDown(): moverá a Mecca una casilla hacia abajo.
   \end{itemize}
  \item Funciones que se usan para que Mecca dispare:
  \begin{itemize}
   \item shootUp(): permitirá disparar hacia arriba.
   \item shootRight(): permitirá disparar hacia la derecha.
   \item shootLeft(): permitirá disparar hacia la izquierda.
   \item shootUp(): permitirá disparar hacia abajo.
  \end{itemize}

  \end{itemize}


  
  \section{Sentencias del lenguaje de programación de pseudocódigo}
  Las sentencias del lenguaje de programación serán:
  \begin{itemize}
   \item Sentencia condicional simple
   \item Sentencia condicional compuesta
   \item Bucle ``mientras''
   \item Bucle ``repetir''
   \item Bucle ``para''
  \end{itemize}
  Dado que las otras órdenes serán en inglés, se ha decidido, para mantener una cierta homogeneidad en este aspecto, que 
  estas sentencias también usen ese idioma.
    \subsection{Sentencia condicional simple}
    \textbf{if} condicion \textbf{then}\\
      sentencias\\
    \textbf{end\_if;}
    \subsection{Sentencia condicional compuesta}
    \textbf{if} condicion \textbf{then}\\
      sentencias\\
    \textbf{else}\\
      sentencias\\
    \textbf{end\_if};
    \subsection{Bucle ``mientras''}
    \textbf{while} condicion \textbf{do}\\
      sentencias\\
    \textbf{end\_while;}\\
    \subsection{Bucle ``repetir''}
    \textbf{repeat}\\ 
    sentencias\\
    \textbf{until} condicion\textbf{;}\\
    
    \subsection{Bucle ``para''}
    \textbf{for} identificador\\
      \textbf{from} expresión numérica 1\\
      \textbf{until} expresión numérica 2\\
      \textbf{step} expresión numérica 3\\
    \textbf{do}\\
      sentencias\\
    \textbf{end\_for;}\\
  
  
  \chapter{Descripción de la gramática asociada al lenguaje definido \label{defgra}}

\chapter{Descripción del intérprete ANTLR construido para la gramática \label{defantlr}}
  \section{Analizador léxico: componentes léxicos}
  El analizador léxico se ha codificado en el fichero \emph{Meccalex.g} y cuenta con la definición de los siguientes \emph{tokens}:
  \begin{itemize}
   \item Modos de utilización del intérprete
   \begin{lstlisting}
    // Modos de utilización del intérprete
    BEGIN_CONF       = "ConfigurationMode";
    END_CONF         = "endConfigurationMode";
    BEGIN_ADV        = "AdventureMode";
    END_ADV          = "endAdventureMode";
   \end{lstlisting}
   
   \item Funciones que pueden realizarse
   \begin{lstlisting}
    //Funciones
    FUNC_LEER = "read";
    FUNC_ESCRIBIR = "write";
    FUNC_SHOWBOARD = "showBoard";
    FUNC_SHOWADVENTURESTATE = "showAdventureState";
    FUNC_SETBOARDSIZE = "setBoardSize";
    FUNC_GETBOARDROWS = "getBoardRows";
    FUNC_GETBOARDCOLUMNS = "getBoardColumns";
    FUNC_GETBOARDSIZE = "getBoardSize";
    FUNC_SETTREASURE = "setTreasure";
    FUNC_GETTOTALTREASURES = "getTotalTreasures";
    FUNC_GETTREASURE = "getTreasure";
    FUNC_SHOWTREASURES = "showTreasures";    
    FUNC_REMOVETREASURE = "removeTreasure";
    FUNC_SETHOLE = "setHole";
    FUNC_REMOVEHOLE = "removeHole";
    FUNC_GETNUMBEROFHOLES = "getNumberOfHoles";
    FUNC_GETHOLE = "getHole";
    FUNC_SHOWHOLES = "showHoles";
    FUNC_SETWUMPUS = "setWumpus";
    FUNC_GETWUMPUS = "getWumpus";
    FUNC_SETSTART = "setStart";
    FUNC_GETSTART = "getStart";
    FUNC_SETEXIT = "setExit";
    FUNC_GETEXIT = "getExit";
    FUNC_GETMECCA = "getMecca";
    FUNC_GETREMAININGTREASURES = "getRemainingTreasures";
    FUNC_SETARROWS = "setArrows";
    FUNC_GETARROWS = "getArrows";
    FUNC_INCARROWS = "incArrows";
    FUNC_DECARROWS = "decArrows";
    FUNC_SHOOTLEFT = "shootLeft";
    FUNC_SHOOTRIGHT = "shootRighT";
    FUNC_SHOOTUP = "shootUp";
    FUNC_SHOOTDOWN = "shootDown";
    FUNC_GOLEFT = "goLeft";
    FUNC_GORIGHT = "goRight";
    FUNC_GOUP = "goUp";
    FUNC_GODOWN = "goDown";
    \end{lstlisting}
    
  \item Palabras reservadas para los tipos de variables
  \begin{lstlisting}
    // Tipos de variables
    TIPO_CADENA   = "String";
    TIPO_NUMERO = "Number";
  \end{lstlisting}
  
  \item Palabras reservadas para bucles y esquemas condicionales
  \begin{lstlisting}
    //Palabras reservadas para bucles y esquemas condicionales
    RES_MIENTRAS     = "while" ;
    RES_HACER        = "do" ;
    RES_REPETIR		 = "repeat";
    RES_HASTA        = "until";
    RES_FIN_MIENTRAS = "end_while" ;
    RES_SI           = "if" ;
    RES_ENTONCES     = "then" ;
    RES_SI_NO        = "else" ;
    RES_FIN_SI       = "end_if" ;
    RES_PARA         = "for" ;
    RES_PASO	     = "step" ;
    RES_DESDE        = "from" ;
    RES_FIN_PARA     = "end_for" ;
  \end{lstlisting}


  \end{itemize}

  \section{Analizador sintáctico}
  El analizador sintáctico está codificado en el fichero \emph{Meccasint.g} y está dividido en las siguientes partes.
    \subsection{Atributos de la clase Meccasint.g} %Hecho por mí
    
    La variable de tipo cadena \emph{mode} es muy importante, ya que es la encargada de controlar en qué modo estamos 
    dentro del intérprete. Los valores que puede contener están definidos en las tres cadenas de tipo estático del 
    comienzo del código mostrado:
    \begin{itemize}
     \item NO\_MODE: será el valor de la cadena \emph{mode} cuando la ejecución no se encuentre en ninguno de los 
     dos modos
     \item CONFIGURATION\_MODE: será el valor de la cadena \emph{mode} cuando la ejecución se encuentre en el modo de 
     configuración
     \item ADVENTURE\_MODE: será el valor de la cadena \emph{mode} cuando la ejecución se encuentre en el modo de 
     aventura
    \end{itemize}
    Más tarde, a la hora de llamar a las funciones dentro de las reglas sintácticas, se comprobará el modo en el que se encuentra 
    la ejecución y se realizarán las acciones pertinentes o se mostrará un mensaje de error en caso de que esa acción no pueda 
    realizarse en ese modo.
    
    La variable de tipo \emph{Board} almacena una instancia de esa clase. El cometido de esta clase es el de almacenar toda 
    la información relativa al tablero y realizar las operaciones y movimientos que sean necesarios en él.
    
    La variable de tipo \emph{TablaSimbolos} almacena una instancia de esa clase. Su objetivo es el de almacenar la tabla de 
    símbolos, con el uso de instancias de la clase \emph{Variable} guardando el nombre del símbolo, el tipo (que puede ser 
    \emph{string} o \emph{number}), y el valor asociado a este símbolo.

    
    \subsection{Símbolos no terminales}
      \subsubsection{Símbolo mecca}
      El símbolo inicial de la gramática es denominado \emph{mecca} y su regla es la siguiente:
      \begin{lstlisting}
      mecca: (instruction)* configuration (instruction)* adventure (instruction)*;
      \end{lstlisting}
      
      Se pueden utilizar tantas instrucciones como se desee dentro y fuera de los métodos (el caso de que puedan o no 
      ejecutarse será controlado por la variable de tipo cadena \emph{mode} como se ha comentado previamente).

      \subsubsection{Símbolo configuration}
      Este símbolo sirve para incluir todas las instrucciones realizadas en el modo de configuración, está delimitado por 
      los símbolos no terminales BEGIN\_CONF y END\_CONF.
      \begin{lstlisting}
      configuration:  BEGIN_CONF {mode = CONFIGURATION_MODE;} (instruction)+ END_CONF;
      \end{lstlisting}
      
      \subsubsection{Símbolo adventure}
      Este símbolo sirve para incluir todas las instrucciones realizadas en el modo de aventura, está delimitado por 
      los símbolos no terminales BEGIN\_ADV y END\_ADV.
      \begin{lstlisting}
      adventure: BEGIN_ADV {if(board.initGame()) mode = ADVENTURE_MODE;} (instruction)+ END_ADV;
      \end{lstlisting}
      
      \subsubsection{Símbolo instruction}
      Este símbolo no terminal es el más extenso de la gramática y es en el que se encuentran todas las funciones que 
      pueden ser llamadas. Contiene dos variables de tipo, valga la redundancia, \emph{Variable} que se utilizan para 
      recoger los resultados de la evaluación de las reglas de tipo \emph{expression} y que son utilizadas dentro de 
      las instrucciones.

    \subsection{Reglas de producción de la gramática}
   
  \section{Análisis semántico}
    \subsection{Atributos: heredados y sintetizados}
    
    \subsection{Funciones auxiliares}
      \subsubsection{Clase Position}
      Se trata de una clase simple cuyo cometido es el de almacenar una posición en el tablero.
      Los atributos con los que cuenta son los siguientes:
      \begin{itemize}
       \item x: es una variable privada de tipo entero entero que almacena la posición x de una instancia de esta clase.
       \item y: es una variable privada de tipo entero entero que almacena la posición y de una instancia de esta clase.
      \end{itemize}
      Los métodos de que dispone son:
      \begin{itemize}
       \item Position(): constructor vacío de la clase que pondrá la posición a (-1,-1).
       \item Position(int newX, int newY): constructor parametrizado que pondrá la posición en los valores que reciba como parámetros.
       \item Position(Position copyPos): constructor de copia de la clase Position.
       \item setX(int newX): función que recibe un valor entero y lo almacena en la variable x.
       \item setY(int newY): función que recibe un valor entero y lo almacena en la variable y.
       \item getX(): función que devuelve un entero con el valor guardado para la posición x.
       \item getY(): función que devuelve un entero con el valor guardado para la posición y.
       \item toString(): función que devuelve una cadena de texto que muestra la posición almacenada en esta instancia.
       \item equals(): función que devuelve true si la posición que se le pasa como parámetro es igual a esta instancia y false en caso contrario.
      \end{itemize}

      \subsubsection{Clase Size}
      Se trata de una clase simple cuyo cometido es el de almacenar el tamaño de algún elemento del tablero (o del propio tablero).
      Sus atributos son:
      \begin{itemize}
       \item x: es una variable privada de tipo entero entero que almacena la anchura de una instancia de esta clase.
       \item y: es una variable privada de tipo entero entero que almacena la altura de una instancia de esta clase.
      \end{itemize}
      Los métodos de que dispone son:
      \begin{itemize}
       \item Size(): constructor vacío de la clase que pondrá el tamaño a (0,0).
       \item Size(int newWidth, int newHeight): constructor parametrizado que pondrá el tamaño en los valores que reciba como parámetros.
       \item Size(Size copyPos): constructor de copia de la clase Size.
       \item setWidth(int newWidth): función que recibe un valor entero y lo almacena en la variable width.
       \item setHeight(int newHeight): función que recibe un valor entero y lo almacena en la variable height.
       \item getWidth(): función que devuelve un entero con el valor guardado para la variable width.
       \item getHeight(): función que devuelve un entero con el valor guardado para la posición height.
       \item toString(): función que devuelve una cadena de texto que muestra la información de tamaños almacenada en esta instancia.
      \end{itemize}

    \subsubsection{Clase Mecca}
    Clase que representa al aventurero del juego. Permite guardar y modificar información sobre el número de flechas que tiene, 
    su posición, y una lista de los tesoros ganados.
    Los atributos con los que cuenta son los siguientes:
    \begin{itemize}
     \item nArrows: variable privada de tipo entero que almacena el número de flechas que tiene Mecca.
     \item position: variable privada de tipo Position que almacena la posición de Mecca en el tablero. 
     \item treasuresWon: variable privada de tipo ArrayList de enteros que almacena los tesoros que han sido conseguidos por Mecca.
    \end{itemize}
    Sus métodos son:
    \begin{itemize}
     \item Mecca(): constructor vacío de la clase. Inicializará la posición con el constructor vacío de Position y 
     el número de flechas a 0, así como los tesoros conseguidos.
     \item getNArrows(): función que devuelve un entero que contiene el número de flechas que le quedan a Mecca.
     \item setNArrows(int newArrows): función que recibe un entero que actualizará el valor de la variable nArrows.
     \item incNarrows(int inc): función que incrementa, en el valor entero que reciba como parámetro, el valor de nArrows.
     \item decNarrows(int inc): función que decrementa, en el valor entero que reciba como parámetro, el valor de nArrows.
     \item incNarrows(): función que incrementa nArrows una unidad.
     \item decNarrows(): función que decrementa nArrows una unidad.
     \item getPos(): función que devuelve un objeto de tipo Position que contiene la posición de Mecca.
     \item setPos(): función que recibe un objeto de tipo Position que será la nueva posición de Mecca.
    \end{itemize}
   \subsubsection{Clase Board}
   Esta clase representa al tablero del juego. Permite guardar y modificar información sobre qué hay en cada casilla y 
   mover a Mecca por todo el tablero. Gestiona todas las acciones que pueden realizarse, además de los distintos resultados 
   que puede tener la partida.
   
   En cuanto a los atirbutos que contiene, en primer lugar, existen unas variables estáticas que contienen los distintos 
   valores que puede albergar una casilla para representar lo que en ella hay dentro del juego. Son las siguientes:
   \begin{lstlisting}
    public static final String WUMPUS = "Wumpus";
	public static final String TREASURE = "Treasure";
	public static final String HOLE = "Hole";
	public static final String BREEZE = "Breeze";
	public static final String SMELL = "Smell";
	public static final String START = "Start";
	public static final String EXIT = "Exit";
	public static final String EMPTY = "Empty"; // No Wumpus, no hole, no treasure, no start, no exit
   \end{lstlisting}
  En el caso de ``Empty'', cabe destacar que una casilla puede ser empty aún conteniendo el olor o la brisa, ya que 
  esos estados son simplemente utilizados para informar a Mecca de las cosas que están cerca, pero no afectan de ninguna 
  manera al estado del juego.
  El resto de atributos que contiene son:
  \begin{itemize}
   \item boardMatrix: es una matriz de ArrayLists de cadenas que contiene, para cada casilla, una lista de cadenas con los 
   elementos que existen en ella.
   \item boardMatrixVisited: es una matriz booleana, del mismo tamaño que el tablero, que controla qué casillas han 
   sido visitadas y cuáles no. Es utilizada para mostrar por pantalla sólo la información de las casillas ya visitadas 
   y que el resto del tablero sea un misterio.
   \item boardSize: atributo de tipo Size que guarda el tamaño del tablero.
   \item mecca: atributo de tipo Mecca que guarda toda la información del aventurero.
   \item wumpusPos: atributo de tipo Position que guarda información sobre la posición del Wumpus.
   \item startPos: atributo de tipo Position que guarda información sobre la posición de la casilla de comienzo.
   \item exitPos: atributo de tipo Position que guarda información sobre la posición de la casilla de salida.
   \item treasuresPos: ArrayList de posiciones que contiene la posición de los tesoros existentes en el tablero.
   \item holesPos: ArrayList de posiciones que contiene la posición de los hoyos existentes en el tablero.
   \item isWumpusAlive: atributo booleano que será true si el Wumpus está vivo y false en caso contrario.
   \item endOfGame: atributo booleano que será true si el juego ha terminado ya y falso en caso contrario.
  \end{itemize}
  
  Los métodos de esta clase son:
  \begin{itemize}
   \item Board(): constructor vacío de la clase Board. Por defecto todos los tamaños y posiciones son inicializados con sus 
   respectivos constructores vacíos.
   \item restartBoard: método que inicializa el tablero colocando todas sus casillas como vacías. 
   \item isEmpty(Position pos): metodo que devuelve verdadero si la casilla que se le pasa como argumento está vacía y falso en 
   caso contrario.
   \item isInsideBoard(Position pos): metodo que devuelve verdadero si la casilla que se le pasa como argumento está dentro 
   del trablero y falso en caso contrario.
   \item writeOnBoard(Position pos, String element): escribe en la posición del tablero que se le pasa como primer parámetro el 
   elemento que recibe como segundo.
   \item removeFromBoard(String element, Position position): borra de la posición del tablero que se le pasa como segundo parámetro 
   el elemento que se le pasa como primero.
   \item readFromBoardVisited(Position pos): función privada que devuelve verdadero si la posición que se le pasa como parámetro 
   ya ha sido visitada por Mecca y falso en caso contrario.
   \item writeOnBoardVisited(Position pos, boolean element): función privada que escribe (en la matriz de casillas visitadas) en la posición 
   que se le pasa como primer parámetro el booleano que se le pasa como segundo parámetro.
   \item getSize(): función que devuelve el tamaño del tablero en una variable de tipo Size.
   \item setSize(): función que recibe como parámetro una variable de tipo Size y hace que este sea el nuevo tamaño 
   del tablero, además de reiniciarlo.
   \item getWumpusPos(): función que devuelve una variable de tipo Position con la posición del Wumpus.
   \item setWumpusPos(Position newPos): función que coloca el Wumpus en la posición que recibe como parámetro. 
   Devolverá true si todo funciona y falso en caso contrario.
   \item removeWumpus(): función que borra al Wumpus del tablero, eliminado también su olor.
   \item getStartPos(): función que devuelve una variable de tipo Position que contiene la posición de la casilla de comienzo.
   \item setStartPos(Position newPos): función que coloca, si es posible, la casilla de salida en la posición que se le pasa como parámetro. 
   Devolverá true si todo va bien y false en caso contrario.
   \item getExitPos(): función que devuelve una variable de tipo Position que contiene la posición de la casilla de salida.
   \item setExitPos(Position newPos): función que coloca, si es posible, la casilla de salida en la posición que se le pasa como parámetro. 
   Devolverá true si todo va bien y false en caso contrario.
   \item setTreasurePos(Position newPos): función que permite añadir tesoros a la lista de tesoros en la posición que se le pase como parámetro.
   Devuelve un entero que contiene la posición en la que se encuentra el tesoro en la lista.
   \item getTreasurePos(int treasureNo): función que devuelve la posición en el trablero del tesoro que se encuentra 
   en la posición de la lista que indica el entero que recibe como parámetro. 
   \item getTreasuresPos(): función que devuelve toda la lista de los tesoros con sus posiciones.
   \item showTreasures(): función que muestra un mensaje por consola que indica la posición de cada tesoro.
   \item editTreasurePos(int treasureNo, Position newPos): función que permite editar la posición del tesoro cuya posición en el vector se le pasa 
   como primer parámetro. La posición que se le pondrá es la indicada por el segundo parámetro.
   \item removeTreasure(int treasureNumber): función que borra de la lista de tesoros aquél cuya posición en esa lista sea la que 
   se le pasa como parámetro.
   \item removeTreasure(Position position): borra de la lista de tesoros el que se encuentre en la posición del tablero que se le pase 
   como parámetro.
   \item setHolePos(Position newPos): función que permite añadir hoyos a la lista de hoyos en la posición que se le pase como parámetro.
   Devuelve un entero que contiene la posición en la que se encuentra el tesoro en la lista.
   \item getHolePos(int holeNo): función que devuelve la posición en el trablero del hoyo que se encuentra 
   en la posición de la lista que indica el entero que recibe como parámetro. 
   \item getHolesPos(): función que devuelve toda la lista de los hoyos con sus posiciones.
   \item editHolePos(int holeNo, Position newPos): función que permite editar la posición del hoyo cuya posición en el vector se le pasa 
   como primer parámetro. La posición que se le pondrá es la indicada por el segundo parámetro.
   \item removeHole(int holeNumber): función que borra de la lista de hoyos aquél cuya posición en esa lista sea la que 
   se le pasa como parámetro.
   \item showHoles(): función que muestra un mensaje por consola que indica la posición de cada hoyo.
   \item getTotalTreasures(): función que devuelve un entero con la cantidad de tesoros existentes.
   \item getNumberOfHoles(): función que devuelve un entero con la cantidad de hoyos existentes.
   \item getMecca(): función que devuelve la instancia de la clase Mecca que se contiene en este tablero.
   \item getMeccaPos(): función que devuelve un objeto de tipo Position que contiene la posición de Mecca.
   \item setMeccaPos(Position newPos): función que recibe un objeto de tipo Position que será la nueva posición de Mecca.
   \item getMeccaNArrows(): función que devuelve un entero que contiene el número de flechas que le quedan a Mecca.
   \item setMeccaNArrows(int nArrows): función que recibe un entero que actualizará el valor de la variable que contiene el número de flechas que le quedan a Mecca.
   \item incMeccaNArrows(int inc): función que incrementa, en el valor entero que reciba como parámetro, el valor del número de flechas que le quedan a Mecca.
   \item decMeccaNArrows(int dec): función que decrementa, en el valor entero que reciba como parámetro, el valor del número de flechas que le quedan a Mecca.
   \item meccaGoUp(): función que mueve a Mecca una casilla hacia arriba.
   \item meccaGoDown(): función que mueve a Mecca una casilla hacia abajo.
   \item meccaGoLeft(): función que mueve a Mecca una casilla hacia la izquierda.
   \item meccaGoRight(): función que mueve a Mecca una casilla hacia la derecha.
   \item meccaShoot(int direction): función booleana que permite a Mecca disparar. Devolverá true si acaba con el Wumpus
   y false en caso contrario. Recibe un entero que indica la dirección de disparo con los siguientes valores:
    \begin{itemize}
     \item 1.- Arriba
     \item 2.- Derecha
     \item 3.- Izquierda
     \item 4.- Abajo
    \end{itemize}
   \item meccaGoPosition(Position position): función privada de tipo booleano que coloca a Mecca en la posición que recibe 
   como parámetro. Devolverá true si el movimiento pudo ser efectuado y falso en caso contrario.
   \item checkMovement(Position position): función privada de tipo booleano que comprueba el resultado de un movimiento de Mecca.
   Devuelve true si puede realizarse y false en caso contrario. Informa por consola del resultado de ese movimiento.
   \item getBoardSize(): función que devuelve un objeto de tipo Size con el tamaño del tablero.
   \item toString(): función que devuelve una cadena que contiene toda la información relativa al tablero.
   \item isWumpusAlive(): función booleana que devuelve true si el Wumpus está vivo y false en caso contrario.
   \item setWumpusAlive(boolean isWumpusAlive): función que recibe un booleano que será el que indique si el Wumpus está vivo o no y lo guarda.
   \item initGame(): función booleana que comprueba si, al entrar en el modo aventura, el mapa definido tiene todo lo necesario 
   para comenzar el juego. En caso afirmativo devuelve true, en caso negativo false.
   \item isGameFinished(): función booleana que devuelve true si el juego ha terminado y false en caso contrario.
   \item checkStart(): función booleana que comprueba si la casilla de comienzo ha sido colocada. 
   Devuelve true en caso de que lo haya sido y false en caso contrario.
   \item checkExit(): función booleana que comprueba si la casilla de salida ha sido colocada. 
   Devuelve true en caso de que lo haya sido y false en caso contrario.
   \item checkWumpus(): función booleana que comprueba si la casilla del Wumpus ha sido colocada. 
   Devuelve true en caso de que lo haya sido y false en caso contrario.
   \item showAdventureState(): fnción que muestra por consola el estado de la aventura. Es decir, muestra todos los 
   elementos que el jugador conoce porque ya ha visitado esas casillas.
   \end{itemize}
   
  \section{Tabla de símbolos}
  En este proyecto, la tabla de símbolos se implementa con una clase que contiene como atributo un ArrayList de objetos 
  de la clase Variable. Por esta razón, va a explicarse primero esa clase y posteriormente se explicará la tabla propiamente 
  dicha.
  
  \subsection{Clase Variable}
  Esta clase, realizada por Luis Del Moral, Juan María Palomo y Profesor, y modificada por los autores de este trabajo 
  contiene lo necesario para representar una   variable de las que se almacenarán en la tabla de símbolos. 
  Sus atributos, todos privados son los siguientes:
  \begin{itemize}
   \item \_nombre: que representa el nombre del identificador.
   \item \_tipo: que representa el tipo del identificador. En esta aplicación concreta los tipos serán number y string.
   \item \_valor: que representa el valor que contiene ese identificador.
  \end{itemize}
  En cuanto a los métodos que contiene, son los siguientes:
  \begin{itemize}
   \item Variable(): constructor vacío que no realiza ninguna operación.
   \item Variable(String nombre,String tipo, String valor): constructor parametrizado que recibe el nombre, tipo y 
   valor de una variable a crear y los guarda en los atributos privados ya comentados.
   \item setNombre(String nombre): método que permite guardar el nombre que se le pasa como parámetro.
   \item setValor(String valor): método que permite guardar el valor que se le pasa como parámetro.
   \item getNombre(): método que devuelve un entero con el nombre de la variable.
   \item getTipo(): método que devuelve un entero con el tipo de la variable.
   \item escribirVariable(): método que escribe por pantalla toda la información relativa a una variable, es decir, nombre, 
   tipo y valor.
   \item isNumber(): función booleana que devuelve true si la variable es de tipo number y false en caso contrario.
   \item isString(): función booleana que devuelve true si la variable es de tipo string y false en caso contrario.
  \end{itemize}

  Una vez definida esta clase, ya se puede entender perfectamente la clase TablaSimbolos.
  \subsection{Clase TablaSimbolos}
  Esta clase, realizada por Luis Del Moral, Juan María Palomo y Profesor, y modificada por los autores de este trabajo, 
  contiene lo necesario para guardar los identificadores, tipos y valores de todas las variables creadas durante la ejecución 
  del intérprete.
  El único método que contiene es un ArrayList privado de objetos de la clase Variable ya explicada.
  Los métodos que implementa son:
  \begin{itemize}
   \item TablaSimbolos(): constructor vacío en el que se inicializa el Arraylist.
   \item insertarSimbolo(Variable nuevoSimbolo): función cuya misión es insertar la variable que recibe
   como parámetro en la tabla de símbolos.
   \item getSimbolo(int indice): devuelve la variable que ocupa la posición señalada por indice.
   \item existeSimbolo(String nombreSimbolo): función que comprueba si existe el símbolo cuyo nombre recibe como 
   parámetro en la tabla de símbolos. Si existe, devuelve el índice que indica la posición del símbolo, en caso contrario, 
   devuelve -1.
   \item eliminarSimbolo(String nombreSimbolo): función que elimina de la tabla de símbolos el identificador cuyo nombre recibe como 
   parámetro.
   \item escribirSimbolos(): función que escribe por pantalla los identificadores de la tabla de símbolos.
  \end{itemize}
  
  \section{Elementos auxiliares para la simulación del entorno diseñado}

\chapter{Modo de obtención del intérprete: descripción del fichero makefile}

  \section{Nombre y descripción de cada uno de los ficheros utilizados}
  
  \section{Modo de generación de intérprete}
  
\chapter{Modo de ejecución del intérprete}

  \section{Interactiva}
  
  \section{A partir de un fichero}
  
\chapter{Ejemplos}
  
  En el presente capítulo van a mostrarse algunos ficheros de ejecución realizados para realizar la fase de pruebas y, 
  además, mostrar las diferentes posibilidades del intérprete desarrollado. Cada uno de los ejemplos intenta enfatizar 
  una de las distintas posibilidades del intéprete. En todos ellos hay comentarios tanto de una línea como de varias.
  Los ejemplos que van a verse son los siguientes:
  \begin{itemize}
   \item Ejemplo 1: Juego Básico
   \item Ejemplo 2: Variables (declaración y uso)
   \item Ejemplo 3: Uso de los esquemas condicionales
   \item Ejemplo 4: Juego avanzado
   \item Ejemplo 5:
  \end{itemize}

  
  \section{Ejemplo 1: Juego Básico}
  
  En este fichero de ejemplo se muestra cómo crear un tablero de juego, 
  realizando las siguientes acciones:
  
  \begin{itemize}
   \item Entrada en el modo de configuración
   \begin{itemize}
    \item Selección del tamaño del tablero
    \item Comprobación de que el comando anterior funcionó, llamando a las funciones 
    encargadas de devolver el valor del número de filas y columnas
    \item Selección de las casillas de comienzo y de salida
    \item Selección de las posiciones de 3 hoyos
    \item Comprobación de las posiciones de los hoyos
    \item Selección del número de flechas que tendrá Mecca
    \item Selección de la posición de un tesoro
    \item Comprobación de la posición del tesoro
    \item Selección de la posición del Wumpus
    \item Salida del modo de configuración
    \end{itemize}
   \item Entrada en el modo aventura
      \begin{itemize}
      \item Realización de movimientos, comprobando que se muestra el 
      olor del Wumpus
      \item Llegada de Mecca a la casilla del tesoro
      \item Llegada de Mecca a la casilla de salida
      \end{itemize}
   \end{itemize}
  
  Código del ejemplo:
  \begin{lstlisting}
    { Test file number 01
    We will show how to create a MeccaBoard 
    and play one game}

    ConfigurationMode   #This is the mode to create the board
    setBoardSize(5,5);  #Size of 5 rows and 5 columns
    getBoardRows();     #Yeah, it works!
    getBoardColumns();  #Again!
    setStart(0,0);      #Down left corner
    setExit(4,4);	    #Up right corner
    setHole(1,1);	    
    setHole(2,2);
    setHole(0,1);
    showHoles();
    setArrows(3);
    setTreasure(4,3);
    showTreasures();
    setWumpus(4,0);
    endConfigurationMode
    {
    This is what the level should look like

	    -	-	-	-	X
	    -	~	~	~	T
	    ~	~	H	~	-
	    H	H	~	~	=
	    +	~	~	=	W

    }
    #Let's start the adventure!!!
    AdventureMode
    goRight();
    goRight();
    goRight();	#It should smell bad right now
    goUp();		#Let's avoid the Wumpus
    goRight();
    goUp();
    goUp();		#You should have found a treasure here
    goUp();		#Winner
    endAdventureMode
  \end{lstlisting}

  
  \section{Ejemplo 2: Variables (declaración y uso)}
  En este ejemplo se muestra cómo se declaran y utilizan las variables. Como puede observarse, éstas pueden ser 
  definidas y utilizadas fuera de los modos de configuración o aventura. Las acciones realizadas son las siguientes:
  \begin{itemize}
   \item Declaración, con asignación de valor, de tres variables numéricas
   \item Declaración, también con asignación de valor, de cuatro variables de tipo cadena
   \item Declaración de una cadena de cara a su impresión por pantalla siéndole asignado el resultado de una operación 
   de suma entre las tres variables de cadena creadas anteriormente
   \item Utilización del comando de impresión por pantalla para mostrar la cadena que ha sido creada en el paso anterior
   \item Asignación a la cadena que ha sido imprimida de valores diferentes y realización de nuevas impresiones
   \item Creación de una nueva variable numérica a la que se le asigna el valor de multiplicar las tres variables numéricas 
   creadas al comienzo del ejemplo
   \item Impresión del valor de la variable que se acaba de crear
  \end{itemize}

  Código del ejemplo:
  \begin{lstlisting}
    { Test file number 02
    We will show how to declare and use variables }

    #Variables can be declared outside the modes
    number jordan:=23;
    number magic:=32;
    number bird:=33;
    string jor:="Jordan";
    string mag:="Magic";
    string bir:="Bird";
    string player_presentation:=" number is ";


    #Now we are going to print them
    string printout := jor+player_presentation+jordan;
    read(printout);

    printout := mag+player_presentation+magic;
    read(printout);

    printout := bir+player_presentation+bird;
    read(printout);

    #Let's do a mathematical operation
    number result:=jordan*magic*bird;
    read(result);
  \end{lstlisting}

  \section{Ejemplo 3: Uso de los esquemas condicionales}
  En este ejemplo se crean tres variables numéricas, se utilizan esquemas condicionales comparando los valores de las mismas 
  y se crean las variables que corresponda dependiendo de estos valores y, posteriormente se muestran por pantalla.
  Los pasos realizados son los siguientes:
  \begin{itemize}
   \item Creación de tres variables numéricas
   \item Esquema condicional en el que se comparan dos valores numéricos y se asigna un valor a una variable de tipo cadena 
   dependiendo del resultado de esa comparación
   \item Impresión de la variable creada
   \item Utilización de dos esquemas condicionales anidados para la declaración de una segunda variable cadena, cuyo valor 
   depende del resultado de la declaración
   \item Impresión de la variable creada
  \end{itemize}
  
  Código del ejemplo:
  \begin{lstlisting}
  { Test file number 03
  We will show how to use conditional schemes }

  #Variables can be declared outside the modes
  number jordan:=23;
  number magic:=32;
  number bird:=33;

  if jordan > magic then
	  string print:=jordan+" is bigger than "+magic;
  else
	  string print:=jordan+" is not bigger than "+magic;
  end_if;

  read(print);

  if jordan < magic then
	  if jordan < bird then
		  string print2:=jordan+" is smaller than "+magic+" and "+bird;
	  end_if;
  end_if;

  read(print2);

  #You can also do it like this

  if jordan < magic _and jordan < bird then
	  string print2:=jordan+" is smaller than "+magic+" and "+bird + " second way ";
  end_if;

  read(print2);

  \end{lstlisting}

  \section{Ejemplo 4: Juego Avanzado}
  
  En este fichero de ejemplo se muestra cómo crear un tablero de juego como el del ejemplo 1 pero haciendo uso 
  de elementos más avanzados como bucles, esquemas condicionales y variables:
  
  \begin{itemize}
   \item Entrada en el modo de configuración
   \begin{itemize}
    \item Creación de las variables de filas y columnas, denominadas rows y cols respectivamente
    \item Asignación de un valor a la variable de filas
    \item Asignación a la variable de columnas del valor contenido en la de filas
    \item Selección del tamaño del tablero usando ambas variables
    \item Comprobación de que el comando anterior funcionó, llamando a las funciones 
    encargadas de devolver el valor del número de filas y columnas
    \item Selección de la posición de la casilla de comienzo, cuyas coordenadas son el resultado de restar las dos variables anteriores
    \item Selección de la posición de la casilla de finalización, cuyas coordenadas son el resultado de restarle uno a las dos variables
    \item Utilización de un esquema condicional para insertar posiciones de hoyos
    \item Comprobación de las posiciones de los hoyos
    \item Utilización de un esquema de tipo ``mientras'' para la selección del número de flechas que tendrá Mecca
    \item Selección de la posición de un tesoro
    \item Comprobación de la posición del tesoro
    \item Selección de la posición del Wumpus
    \item Salida del modo de configuración
    \end{itemize}
   \item Entrada en el modo aventura
      \begin{itemize}
      \item Utilización de un bucle del tipo ``para'' con el objetivo de mover a Mecca tres casillas a la derecha
      \item Realización de dos movimientos directamente
      \item Creación de una variable de tipo numérico que nos servirá en el bucle siguiente
      \item Utilización de un bucle ``repetir hasta'' que moverá a Mecca tres veces hacia arriba que producirá los siguientes eventos:
      \begin{itemize}
      \item Llegada de Mecca a la casilla del tesoro
      \item Llegada de Mecca a la casilla de salida
      \end{itemize}
      \item Creación de dos variables de tipo cadena para mostrar un mensaje de despedida
      \item Impresión en consola de las dos cadenas recién creadas usando \emph{read}
      \item Utilización de \emph{read} para imprimir por pantalla una cadena directamente insertada por el usuario
      \end{itemize}
   \end{itemize}
  
  Código del ejemplo:
  \begin{lstlisting}
{ Test file number 04
We will show how to create a MeccaBoard 
and play one game while we use all kind of loops}

ConfigurationMode   #This is the mode to create the board
number rows, cols;
rows:=5;
cols:=rows;
setBoardSize(rows,cols);  #Size of 5 rows and 5 columns
getBoardRows();     #Yeah, it works!
getBoardColumns();  #Again!
setStart(cols-rows,rows-cols);      #Down left corner
setExit(cols-1,rows-1);	    #Up right corner
if cols == rows then    #Conditional
setHole(1,1);	    
setHole(2,2);
setHole(0,1);
else
setHole(1,1);
end_if;
showHoles();
while rows > 2 do    #While Loop
incArrows(1);
rows:=rows-1;
end_while;
setTreasure(4,3);
showTreasures();
setWumpus(4,0);
endConfigurationMode
{
This is what the level should look like

	-	-	-	-	X
	-	~	~	~	T
	~	~	H	~	-
	H	H	~	~	=
	+	~	~	=	W

}
#Let's start the adventure!!!
AdventureMode
for i from 0 until 3 step 1
do
goRight();
end_for;	#We are using a for loop for moving 3 times
goUp();		
goRight();
number limit:=0;
repeat         #Repeat-until loop
goUp();
limit:=limit+1;
until limit==3;
string final:= "/************************/";
string final2:= "/*****see you soon!******/";
write(final);
write(final2);
write("/************************/");
endAdventureMode
  \end{lstlisting}


%\appendix
%\chapter{Nombre del anexo i \label{anexoi}}
%Descomenta la línea siguiente y sustituye anexoi.tex por el archivo con tu primer anexo
%\input{anexoi.tex}

%\chapter{Nombre del anexo ii \label{anexoii}}
%Descomenta la línea siguiente para el segundo anexo, y así sucesivamente
%\input{anexoii.tex}

%\newpage



%%%%%%%%%%%%%%%%%%%%%%%%%%%%%%%%%%%%%%%%%
%  Se acabó el documento, ahora quedan  %
%  unos pequeños detalles               %
%%%%%%%%%%%%%%%%%%%%%%%%%%%%%%%%%%%%%%%%%

%Añade la bibliografía al índice
\addcontentsline{toc}{chapter}{Bibliografía}

%Define el estilo de la bibliografía
\bibliographystyle{unsrt} 

%La siguiente línea hace referencia al archivo de la bibliografía. 
%Aquí utilizamos el archivo de ejemplo `archivo.bib` 
%que deberías haber descargado junto con esta plantilla.
%\bibliography{archivo} 
%Recuerda que hay que compilar dos veces

\begin{thebibliography}{8}
\bibitem{Escarabajo} Edgar Allan Poe. El escarabajo de oro. Editorial: Losada. Año 2010. I.S.B.N : 9789500397124.
\bibitem{Nuez} Stephen W. Hawking. El universo en una cáscara de nuez. Editorial: Círculo de lectores. Año 2005. I.S.B.N : 84-843-2293-9.
\bibitem{Historia}  Stephen W. Hawking. Brevísima historia del tiempo. Editorial: Círculo de lectores. Año 2005. I.S.B.N : 84-672-1469-4.
\bibitem{Presentation} Stephen Bartlett. Lecture 5 on Quantum Computing. NITP Summer School, Adelaide, Australia. Año: 2003. Dirección web: \url{http://www.cs.duke.edu/~reif/courses/complectures/AltModelsComp/QuantCrypt/QuantCryptOverview/Bartlett/QuantCryptOverviewBartlett.pdf}. Fecha de última consulta: 13 de Mayo de 2012.
\bibitem{QKD} Charles H. Bennett, Gilles Bras Sard and Artur K. Ekert. Quantum Cryptography. Editorial: Scientific American. Año: 1992. Dirección web: \url{http://www.dhushara.com/book/quantcos/aq/qcrypt.htm}. Fecha de última consulta: 13 de Mayo de 2012.
\bibitem{PFC} Jesús Martínez Mateo. PFC: Criptografía cuántica aplicada. Universidad Politécnica de Madrid. Año: 2008. Dirección web: \url{http://oa.upm.es/1298/1/PFC_JESUS_MARTINEZ_MATEO.pdf}.  Fecha de última consulta: 13 de Mayo de 2012.
\bibitem{Noticia} Nuevo impulso a la criptografía cuántica. Universidad Politécnica de Madrid. Año: 2012. Dirección web: \url{http://www.fi.upm.es/?id=tablon&acciongt=consulta1&idet=1068}. Fecha de última consulta: 13 de Mayo de 2012. 
\bibitem{PHD} Jesús Martínez Mateo. Tésis Doctoral: Reconciliación Eficiente de Información para la Distribución Cuántica de Claves. Universidad Politécnica de Madrid. Año: 2011. Dirección web: \url{http://oa.upm.es/9717/1/Jesus_Martinez_Mateo.pdf}. Fecha de última consulta: 13 de Mayo de 2012.
\end{thebibliography}

\end{document} %Adios